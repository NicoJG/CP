% Warnung, falls nochmal kompiliert werden muss
\usepackage[aux]{rerunfilecheck}
% Fonteinstellungen
\usepackage{fontspec} % Latin Modern Fonts werden automatisch geladen
% unverzichtbare Mathe-Befehle
\usepackage{amsmath}
% viele Mathe-Symbole
\usepackage{amssymb}
% Erweiterungen für amsmath
\usepackage{mathtools}
% Math standards
\usepackage[
  math-style=ISO,    % ┐
  bold-style=ISO,    % │
  sans-style=italic, % │ ISO-Standard folgen
  nabla=upright,     % │
  partial=upright,   % ┘
  warnings-off={           % ┐
    mathtools-colon,       % │ unnötige Warnungen ausschalten
    mathtools-overbracket, % │
  },                       % ┘
]{unicode-math}
% traditionelle Fonts für Mathematik
\setmathfont{Latin Modern Math}
% SI Zahlen und Einheiten
\usepackage[
  locale=DE,                   % deutsche Einstellungen
  separate-uncertainty=true,   % immer Unsicherheit mit \pm
  per-mode=symbol-or-fraction, % / in inline math, fraction in display math
]{siunitx}
% Hyperlinks im Dokument
\usepackage[
  english,
  unicode,        % Unicode in PDF-Attributen erlauben
  pdfusetitle,    % Titel, Autoren und Datum als PDF-Attribute
  pdfcreator={},  % ┐ PDF-Attribute säubern
  pdfproducer={}, % ┘
]{hyperref}
% Floats innerhalb einer Section halten
\usepackage[
  section, % Floats innerhalb der Section halten
  below,   % unterhalb der Section aber auf der selben Seite ist ok
]{placeins}
% Trennung von Wörtern mit Strichen
\usepackage[shortcuts]{extdash}
% Verbesserungen am Schriftbild
\usepackage{microtype}
% schöne Tabellen
\usepackage{booktabs}