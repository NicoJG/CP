\section{Linear regression}

\begin{table}
    \centering
    \caption{data points $(x,y)$}
    \label{tab:data_points}
    \begin{tabular}{c c c c c c c c c c c}
        \toprule
        x & 0 & 2.5 & -6.3 & 4 & -3.2 & 5.3 & 10.1 & 9.5 & -5.4 & 12.7 \\
        y & 4 & 4.3 & -3.9 & 6.5 & 0.7 & 8.6 & 13 & 9.9 & -3.6 & 15.1 \\
        \bottomrule
    \end{tabular}
\end{table}

% a) and b)
\subsection{Problem formulation}

% TODO: overdetermined lse to square matrix

To calculate a linear regression of the data points with
\begin{equation}
    y(x) = m \cdot x + n \, ,
\end{equation}
the squared error
\begin{equation}
    R = \sum_i (m \cdot x_i + n - y_i)^2
\end{equation}
has to be minimized.
$R$ cannot have a local maximum so it is sufficient to calculate
\begin{equation}
    \vec{\nabla}_{m,n} R = 
    \begin{pmatrix}
        \frac{\partial R}{\partial m} \\
        \frac{\partial R}{\partial n}
    \end{pmatrix}
    = 0 \, .
\end{equation}
\begin{align*}
    & \vec{\nabla}_{m,n} R = \sum_i 
    \begin{pmatrix}
        2 m x_i^2 + 2 n x_i - 2 x_i y_i \\
        2 n + 2 m x_i - 2 y_i
    \end{pmatrix} \\
    \Leftrightarrow & \sum_i 
    \begin{pmatrix}
        x_i^2 m + x_i n \\
        x_i m + n
    \end{pmatrix} = \sum_i 
    \begin{pmatrix}
        x_i y_i \\
        y_i
    \end{pmatrix} \\
    \Leftrightarrow & \left( \sum_i 
    \begin{pmatrix}
        x_i^2 & x_i \\
        x_i & 1
    \end{pmatrix} \right) 
    \begin{pmatrix}
        m \\
        n
    \end{pmatrix} = \sum_i 
    \begin{pmatrix}
        x_i y_i \\
        y_i
    \end{pmatrix}
\end{align*}
So the problem can be formulated in the form
\begin{equation}
    \label{eq:system}
    \symbf{A} \cdot \vec{m} = \vec{b} \, 
\end{equation}
with
\begin{align*}
    \symbf{A} &= 
    \left( \sum_i 
    \begin{pmatrix}
        x_i^2 & x_i \\
        x_i & 1
    \end{pmatrix} \right) \\
    \vec{m} &=
    \begin{pmatrix}
        m \\
        n
    \end{pmatrix} \\
    \vec{b} &=
    \begin{pmatrix}
        x_i y_i \\
        y_i
    \end{pmatrix} \, .
\end{align*}

% c) and d)
\subsection{Carrying out the linear regression}

The given data points in \autoref{tab:data_points} yield the following:

\begin{align*}
    \symbf{A} &=
    \begin{pmatrix}
        482.98 & 29.2 \\
        29.2 & 10
    \end{pmatrix}\\
    \vec{b} &=
    \begin{pmatrix}
        541.22 \\
        54.6
    \end{pmatrix}
\end{align*}

Now the LU decomposition yields
\begin{align*}
    \symbf{P} &=
    \begin{pmatrix}
        1 & 0 \\
        0 & 1
    \end{pmatrix} \\
    \symbf{L} &=
    \begin{pmatrix}
        1 & 0 \\
        0.060458 & 1
    \end{pmatrix} \\
    \symbf{U} &=
    \begin{pmatrix}
        482.98 & 29.2 \\
        0 & 8.23463
    \end{pmatrix} \, .
\end{align*}

And solving the system of linear equations \autoref{eq:system} yields
\begin{equation}
    \vec{m} = 
    \begin{pmatrix}
        m \\ 
        n
    \end{pmatrix}
    =
    \begin{pmatrix}
        0.959951 \\ 
        2.65694
    \end{pmatrix} \, .
\end{equation}

Using these parameters $m$ and $n$ to plot a linear regression yields \autoref{fig:plot_exercise2}.

\begin{figure}
    \centering
    \includegraphics[width=0.6\textwidth]{../code/build/plot_exercise2.pdf}
    \caption{Plot of the linear regression}
    \label{fig:plot_exercise2}
\end{figure}