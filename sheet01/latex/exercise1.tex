\section{Change of basis with LU decomposition}

Primitive lattice vectors:
\begin{align}
    \vec{a}_1 &= 
    \begin{pmatrix}
        \frac{1}{2} \\[2pt] \frac{\sqrt{3}}{2} \\[2pt] 0
    \end{pmatrix};
    &
    \vec{a}_2 &=
    \begin{pmatrix}
        -\frac{1}{2} \\[2pt] \frac{\sqrt{3}}{2} \\[2pt] 0
    \end{pmatrix};
    &
    \vec{a}_3 &=
    \begin{pmatrix}
        0 \\
        0 \\
        1
    \end{pmatrix}
\end{align}

% subsections with a) b) c) ...
\renewcommand\thesubsection{\thesection.\alph{subsection})}

% a)
\subsection{What kind of crystal system is represented?}

hcp

% b)
\subsection{Defect at $\vec{b} = (2,0,2)^T$}

Coordinate transformation matrix to $\{\vec{a}_1,\vec{a}_2,\vec{a}_3\}$:
\begin{equation}
    \symbf{A} = (\vec{a}_1,\vec{a}_2,\vec{a}_3) =
    \begin{pmatrix}
        0.5 & -0.5 & 0 \\
        0.866 & 0.866 & 0 \\
        0 & 0 & 1
    \end{pmatrix}
\end{equation}

With the C++ module "Eigen::PartialPivLU" the LU decomposition yields
\begin{align}
    \symbf{P} &= 
    \begin{pmatrix}
        0 & 1 & 0 \\
        1 & 0 & 0 \\
        0 & 0 & 1 
    \end{pmatrix} \\
    \symbf{L} &= 
    \begin{pmatrix}
        1 & 0 & 0 \\
        0.577 & 1 & 0 \\
        0 & 0 & 1 
    \end{pmatrix} \\ \text{and} \\
    \symbf{U} &= 
    \begin{pmatrix}
        0.866 & 0.866 & 0 \\
        0 & -1 & 0 \\
        0 & 0 & 1 
    \end{pmatrix} .
\end{align}

Solving $\symbf{A} \cdot \vec{x} = \vec{b}$ yields 
\begin{equation}
    \vec{x} = 
    \begin{pmatrix*}[r]
        2 \\ -2 \\ 2
    \end{pmatrix*} .
\end{equation}

% c)
\subsection{Defect at $\vec{c} = (1,2\sqrt{3},3)^T$}

You can reuse the LU decomposition which has a complexity of $O(N^3)$.
You have to inversly solve the SLE which has a complexity of $O(N^2)$.

Solving $\symbf{A} \cdot \vec{y} = \vec{c}$ yields
\begin{equation}
    \vec{y} = 
    \begin{pmatrix}
        3 \\ 1 \\ 3
    \end{pmatrix} .
\end{equation}

% d)
\subsection{LU decomposition for the basis $\{\vec{a}_3,\vec{a}_2,\vec{a}_1\}$}

With the C++ module "Eigen::PartialPivLU" the LU decomposition of
\begin{equation}
    \symbf{A}_2 = (\vec{a}_3,\vec{a}_2,\vec{a}_1) =
    \begin{pmatrix}
        0 & -0.5 & 0.5 \\
        0 & 0.866 & 0.866 \\
        1 & 0 & 0 \\
    \end{pmatrix}
\end{equation}
yields
\begin{align}
    \symbf{P}_2 &= 
    \begin{pmatrix}
        0 & 0 & 1 \\
        0 & 1 & 0 \\
        1 & 0 & 0 
    \end{pmatrix} \\
    \symbf{L}_2 &=
    \begin{pmatrix}
        1 & 0 & 0 \\
        0 & 1 & 0 \\
        0 & -0.577 & 1
    \end{pmatrix} \\
    \symbf{U}_2 &=
    \begin{pmatrix}
        1 & 0 & 0 \\
        0 & 0.866 & 0.866 \\
        0 & 0 & 1
    \end{pmatrix} .
\end{align}

The numeric values stay the same as in b) but the position of these values change their position.

% e)
\subsection{How does the problem simplify when the primitive lattice vectors are pairwise orthogonal?}

When the matrix $\symbf{A}$ is orthogonal, its inverse is just the transposed matrix
and the LU decomposition is not necessary.
One can simply calculate $\vec{x} = \symbf{A}^{-1} \cdot \vec{b}$.