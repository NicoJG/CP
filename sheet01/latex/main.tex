\documentclass{scrartcl}

% Warnung, falls nochmal kompiliert werden muss
\usepackage[aux]{rerunfilecheck}
% Fonteinstellungen
\usepackage{fontspec} % Latin Modern Fonts werden automatisch geladen
% unverzichtbare Mathe-Befehle
\usepackage{amsmath}
% viele Mathe-Symbole
\usepackage{amssymb}
% Erweiterungen für amsmath
\usepackage{mathtools}
% Math standards
\usepackage[
  math-style=ISO,    % ┐
  bold-style=ISO,    % │
  sans-style=italic, % │ ISO-Standard folgen
  nabla=upright,     % │
  partial=upright,   % ┘
  warnings-off={           % ┐
    mathtools-colon,       % │ unnötige Warnungen ausschalten
    mathtools-overbracket, % │
  },                       % ┘
]{unicode-math}
% traditionelle Fonts für Mathematik
\setmathfont{Latin Modern Math}
% SI Zahlen und Einheiten
\usepackage[
  locale=DE,                   % deutsche Einstellungen
  separate-uncertainty=true,   % immer Unsicherheit mit \pm
  per-mode=symbol-or-fraction, % / in inline math, fraction in display math
]{siunitx}
% Hyperlinks im Dokument
\usepackage[
  german,
  unicode,        % Unicode in PDF-Attributen erlauben
  pdfusetitle,    % Titel, Autoren und Datum als PDF-Attribute
  pdfcreator={},  % ┐ PDF-Attribute säubern
  pdfproducer={}, % ┘
]{hyperref}
% Trennung von Wörtern mit Strichen
\usepackage[shortcuts]{extdash}
% Verbesserungen am Schriftbild
\usepackage{microtype}
% schöne Tabellen
\usepackage{booktabs}

\subject{Computational Physics\\2021 - TU Dortmund}
\title{Solutions to Exercise Sheet 1}
\author{
    Nico Guth  \\
    \href{mailto:nico.guth@tu-dortmund.de}{nico.guth@tu-dortmund.de}
	\and
	AUTOR B\\
    \href{mailto:authorB@tu-dortmund.de}{authorB@tu-dortmund.de}
	\and
	AUTOR C\\
    \href{mailto:authorC@tu-dortmund.de}{authorC@tu-dortmund.de}
	}

\date{\today}

\begin{document}

\maketitle

\setcounter{section}{-1}
\section{Comprehension questions}

\subsection{What are the possibilities to solve systems of linear equations (with unique solution)?}

System of linear equations: $\symbf{A} \overrightarrow{x}=\overrightarrow{b}$

\begin{itemize}
    \item Calculate the inverse matrix $\Rightarrow \overrightarrow{x} = \symbf{A}^{-1} \overrightarrow{b}$
    \item Gauß algorithm
    \item LU decomposition: $\symbf{A} = \symbf{P} \cdot \symbf{L} \cdot \symbf{U}$ \\
        $\symbf{L}$ is a lower triangular matrix and $\symbf{U}$ is an upper triangular matrix. \\
        $\symbf{P}$ is the "pivoting matrix" and satisfies $\symbf{P}^T=\symbf{P}^{-1}$. \\
        $\Rightarrow ( \symbf{P} \cdot \underbrace{ ( \symbf{L} \cdot \overbrace{ ( \symbf{U} \cdot \overrightarrow{x} ) }^{\overrightarrow{y}\coloneqq} ) }_{\overrightarrow{z} \coloneqq} ) = \overrightarrow{b}$ \\
        Then calculate $\overrightarrow{z} = \symbf{P}^T \cdot \overrightarrow{b}$ 
        and then solve the systems of linear equations:  \\
        $\symbf{L}\cdot\overrightarrow{y}=\overrightarrow{z}$ and $\symbf{U}\cdot\overrightarrow{x} = \overrightarrow{y}$
    \item If the matrix $\symbf{A}$ is not a square matrix, multiply the system of equations with $\symbf{A}^T$ from the left. \\
        Now $\tilde{\symbf{A}} \coloneqq \symbf{A}^T \cdot \symbf{A}$ is a square matrix.
  \end{itemize}

\subsection{Why is a pivoting strategy necessary in general?}

??


\section{Conclusions}\label{conclusions}
There is no longer \LaTeX{} example which was written by \cite{doe}.


\begin{thebibliography}{9}
\bibitem[Doe]{doe} \emph{First and last \LaTeX{} example.},
John Doe 50 B.C. 
\end{thebibliography}

\end{document}