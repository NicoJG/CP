\section{The BFGS Method}
 The BFGS method was implemented according by the lecture. 
 The calculation of the stepsize was done by the backtracking linesearch, taken from the german wikipedia article for the gradient descent method\footnote{https://de.wikipedia.org/wiki/Gradientenverfahren} because our calculation of the stepsize from the last sheet was poorly implemented.
 The three different ways to calculate $C_0$ are abbreviated as $C_a,C_b$ and $C_c$ in the following table, with the index for the corresponding calculations on the sheet.
 \begin{table}[h]
    \centering
    \caption{Minima of the Rosenbrock and iteration count of the BFGS method calculated with the different starting matrices.}
    \label{tab:freq}
    \begin{tabular}{cccc}
        \toprule
        $C_0$ & iteration count & $x_1$ & $x_2$\\
        \midrule
        $C_a$ & $23$ & $0.333428$ & $0.111175$\\
        $C_b$ & $21$ & $0.333331$ & $0.111107$\\
        $C_c$ & $23$ & $0.333171$ & $0.111001$\\
    \end{tabular}
 \end{table}
 The diagonal matrix, on whose diagonals the inverse diagonal elements of the Hessian matrix are located, took the fewest iteration steps.
 All three initial matrices have returned minima in the same order of magnitude as the result of the BFGS method.
