\section{Runge-Kutta method}
In this task, the Runge-Kutta method of 4th order is used to solve Newton's equation of motion for a particle. The force field is given by the force field of the harmonic oscillator
\begin{align}
    \vec{F}\left( \vec{r} \right) = - \vec{r} m \omega^2
\end{align}
and the Newton equation is given by:
\begin{align}
   m \ddot{\vec{r}} = \vec{F}.
\end{align}
This second-order ODE can be rewritten as a first-order ODE for the vector
\begin{align}
    \vec{y} = \begin{pmatrix}
        \vec{r} \\
        \vec{p}
    \end{pmatrix}
\end{align} 
and the equations of motion:
\begin{align}
    \dot{\vec{r}} &= \vec{v} \\
    \dot{\vec{v}} &=  \vec{F}\left( \vec{r} \right) / m.
\end{align}
The Runge-Kutta method of 4th order, without the error, is given by
\begin{align}
    \vec{y}_{n+1} = \vec{y}_n + \frac{1}{6} \left[ \vec{k}_1 + 2\vec{k}_2 + 2\vec{k}_3 + \vec{k}_4 \right].
\end{align}

\subsection{a)}
The Runge-Kutta method is used, to solve the ODE for the initial conditions 
\begin{align}
    \vec{r}(0) &= \text{arbitrary} \\
    \vec{v}(0) &= \vec{0}
\end{align}
exemplarily for one period $T=2\pi \sqrt{\frac{m}{k}} = 2\pi$. Fig.~\ref{fig:ex02_1_2d} shows the harmonic oscillation of the individual coordinates and velocities and Fig.~\ref{fig:ex02_a_1_3d} shows a 3d-plot of the oscillation. Since there is no initial velocity, the particle is moving on a straight line.

\begin{figure}[H]
    \centering
    \includegraphics[width=\textwidth]{../code/build/ex02_a_1_2d.pdf}
    \caption{Oscillation for each component.}
    \label{fig:ex02_1_2d}
\end{figure}

\begin{figure}[H]
    \centering
    \includegraphics[width=\textwidth]{../code/build/ex02_a_1_3d.pdf}
    \caption{Oscillation in 3d.}
    \label{fig:ex02_a_1_3d}
\end{figure}
\noindent If the particle has an initial velocity, that is not parallel to the $\vec{r}$-vector, its individual components also oscillate. This time the particle is not moving on a straight line through the center, but its motion describes a circle aroung the center. Fig.~\ref{fig:ex02_a_2} shows a 3d-Plot of the particles motion.

\begin{figure}[H]
    \centering
    \includegraphics[width=\textwidth]{../code/build/ex02_a_2.pdf}
    \caption{Oscillation in 3d.}
    \label{fig:ex02_a_2}
\end{figure}

\subsection{b)}
In this task, the step size is determined, so that the oscillator satisfies a tolerance limit $\epsilon = |\vec{r}_0  - \vec{r}_i| < 10^{-5}$ after $i=10$ oscillations. For an arbitrary choice of the initial vectors, the step size is given by $h = 1.3 \cdot 10^{-5}$, which yields a relative deviation of $\epsilon = 5.1\cdot 10^{-6}$.

\subsection{c)}
In this task, the total energy is calculated and visualized for a time period of $t=20$. For the sake of simplicity, the starting vectors are chosen to be:
\begin{align}
    \vec{r} &= (1,1,1)^{\text{T}} \\
    \vec{v} &= (0,0,0)^{\text{T}}.
\end{align}
\noindent Fig.~\ref{fig:ex02_c} shows the total energy plotted against the kinetic and potential Energy of the particle. 

\begin{figure}[H]
    \centering
    \includegraphics[width=\textwidth]{../code/build/ex02_c.pdf}
    \caption{Oscillation in 3d.}
    \label{fig:ex02_c}
\end{figure}
