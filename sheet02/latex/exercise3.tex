\section{Construction of an orthonormal basis using SVD}

Given are ten four dimensional vectors $\vec{a}_i$ which are put into the columns of the $\underline{A}$ matrix

\begin{equation*}
    \underline{A} =
    \begin{pmatrix}
        \csvreader[no head,column count=10,late after line=\\]{../code/build/exercise3_A.csv}{}{\csvlinetotablerow}
    \end{pmatrix} \, .
\end{equation*}

Now the singular value decomposition is performed using the C++ module \\
\mbox{"Eigen::BDCSVD"} and yields

\begin{align*}
    \underline{U} &=
    \begin{pmatrix}
        \csvreader[no head,column count=4,late after line=\\]{../code/build/exercise3_U.csv}{}{\csvlinetotablerow}
    \end{pmatrix} \, , \\
    \vec{W} &=
    \begin{pmatrix}
        \csvreader[no head,column count=1,late after line=\\]{../code/build/exercise3_W.csv}{}{\csvlinetotablerow}
    \end{pmatrix}
\end{align*}
and
\begin{equation*}
    \underline{V} =
    \begin{pmatrix}
        \csvreader[no head,column count=10,late after line=\\]{../code/build/exercise3_V.csv}{}{\csvlinetotablerow}
    \end{pmatrix}
\end{equation*}

The columns of $\underline{U}$ are the orthonormal basis vectors of the space spanned by the columns of $\underline{A}$.