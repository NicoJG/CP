\setcounter{section}{-1}
\section{Comprehension questions}
\subsection{Advantages and disandvantages of MD simulation}
A great advantage is the reproducibility of a simulated system.
In addition, extreme conditions can be generated, which are difficult to implement in the laboratory.
The classical MD simulation does not consider electrical effects and chemical reactions.
The Verlet algorithm is also not very accurate.
\subsection{Reasons to use the Verlet algorithm}
The algorithms from the previous sheet are very computationally intensive for N particle systems.
The verlet algorithm is used because it fulfills the requirement of as few computations of $\vec{f}$ as possible.
\subsection{Boundary conditions}

The boundary conditions can be set via an external repulsive wall potential or via periodic boundary conditions. 
The periodic boundary conditions are mostly preferred, because they allow to simulate a quasi infinite system.
Two problems arise here. A particle which moves out of the image has to be inserted correctly on the other side. When calculating the pair interaction it has to be taken care that a particle $i$ interacts not only with another particle $j$ but also with all periodic images of the particle $j$.
This is difficult to realize for long range potentials like the Coulomb potential. 
