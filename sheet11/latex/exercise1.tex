\section{Monte Carlo Simulation of a single Spin}

The Metropolis algorithm is used to simulate a single spin $\sigma = \pm 1$ with the energy
\begin{equation}
\mathcal{H} = - \sigma H 
\end{equation}
in the external magnetic field $H$.
The Metropolis algorithm is implemented according by the script:
\begin{itemize}
\item 1. draw at random if the spin is either 1 or -1
\item 2. calculate $\Delta E = \mathcal{H}(j) - \mathcal{H}(i)$
    \begin{itemize}
        \item if $\Delta E < 0$ accept the result
        \item if $\Delta E > 0$
            \begin{itemize}
            \item draw a uniform distributed number $p \in [0,1]$
            \item if $p < e^{-\beta \Delta E}$ accept the result
            \end{itemize}
    \end{itemize}
\item 3. if the result was accepted we have a new state otherwise we keep the old state    
\end{itemize}
This is repeated for $N = 10^5$ steps for $10^4$ values of $H \in [-5,5]$.
The numerical calculation of $m = <s_n> $ is plotted  against $H$ as well as the analytical value of $m = \text{tanh}(\beta H)$ in Figure\ref{fig:1}.

\begin{figure}[h]
    \centering
    \includegraphics[width=\textwidth]{../code/build/ex01_plot.pdf}
    \label{fig:1}
\end{figure}
