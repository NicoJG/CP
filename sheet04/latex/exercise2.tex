\section{Exercise 2: Multidimensional integrals in electrostatics}
The electrostatic potential alongside the $x-$axis is given by
\begin{align}
    \phi = \frac{1}{4 \pi \epsilon_0} \int \text{d}x' \int \text{d}y' \int \text{d}y' \frac{\rho(x', y', z')}{\sqrt{(x - x')^2 + y'^2 + z'^2}}
\end{align}
and with the substition $\bar{x}_i = \frac{x'_i}{a}$ and $\text{d}x'_i=a\text{d}\bar{x}_i$ you get 
\begin{align}
    \phi &= \frac{a^2}{4 \pi \epsilon_0} \int_{-1}^1 \text{d}\bar{x} \int_{-1}^1 \text{d}\bar{y} \int_{-1}^1 \text{d}\bar{z} \frac{\rho(\bar{x}, \bar{y}, \bar{z})}{\sqrt{(\frac{x}{a} - \bar{x})^2 + \bar{y}^2 + \bar{z}^2}}.
    % &= \phi_0 \int_{-1}^1 \text{d}\bar{x} \int_{-1}^1 \text{d}\bar{y} \int_{-1}^1 \text{d}\bar{z} \frac{1}{\left| \vec{r} - \vec{\bar{r}} \right|} 
\end{align}
To calculate the electrostatic potential at a given point $\vec{r}$ we assume, that the point is at a large distance $\vec{r}'$ from the charge distribution. With help of the binomial expansion of the Coulomb potential, the potential $\phi$ can be expanded to
\begin{align}
    \phi(\vec{r}) = \phi_{\text{mon}}(\vec{r}) + \phi_{\text{dip}}(\vec{r}) +\phi_{\text{quad}}(\vec{r}) +...
\end{align}
with the monopole and dipole moments:
\begin{align}
    \phi_{\text{mon}}(\vec{r}) &= \frac{1}{4\pi \epsilon_0 r} \int_{V'} \rho(\vec{r}') \text{d}V' \\
    \phi_{\text{dip}}(\vec{r}) &= \frac{1}{4\pi \epsilon_0 r^2} \int_{V'} \rho(\vec{r}') ( \hat{\vec{r}} \cdot \vec{r}' ) \text{d}V', \, \hat{\vec{r}} = \frac{\vec{r}}{r}.
\end{align}

\subsection{Homogeneous charge distribution}
The homogeneous charge distribution is given by
\begin{align}
    \rho_1 (x,y,z) = \begin{cases}
        \rho_0, \, |x| < a, |y| < a, |z| < a \\
        0, \, \text{else} 
    \end{cases}
\end{align}
which yields the monopole moment
\begin{align}
    \phi_{\text{mon}}(r) = \frac{a^2}{4 \pi \epsilon_0} \frac{1}{r} \int_{-1}^1 \text{d}\bar{x} \int_{-1}^1 \text{d}\bar{y} \int_{-1}^1 \text{d}\bar{z} \rho_0.
\end{align}
With
\begin{align}
    \vec{r} = 
    \begin{pmatrix}
    \frac{x}{a} \\
    0 \\
    0    
    \end{pmatrix} \, \, \text{and} \,\, \hat{\vec{r}}\begin{pmatrix}
        \bar{x} \\
        -\bar{y} \\
        -\bar{z}
    \end{pmatrix}
\end{align}
it follows
\begin{align}
    \phi_{\text{mon}}(x) = \frac{a^2 \rho_0}{4 \pi \epsilon_0} \frac{8}{|x|} \overset{x > 0}{=} \phi_0 \frac{8}{x}.
\end{align}
For values $\frac{x}{a} \in 0.1\cdot\left\{0, ..., 80  \right\} $, Fig.\ref{fig:ex02_a} shows the potential inside, the potential outside and the analytically estimated potenial.  For the values  $\frac{x}{a} \in 0.1\cdot\left\{0, ..., 10  \right\}$ the denominator gets close to zero, for $n \rightarrow 10$. The easiest way to solve this problem is, to add a small value to the denominator.

\begin{figure}[h]
    \centering
    \includegraphics[width=\textwidth]{../code/build/ex02_a.pdf}
    \caption{Potential inside and outside the cube and the analytically estimated potential for $\rho_1$.}
    \label{fig:ex02_a}
\end{figure}

\subsection{Different charge distribution}
For the charge distribution 
\begin{align}
    \rho_2 (x,y,z) = \begin{cases}
        \rho_0 \frac{x}{a}, \, |x| < a, |y| < a, |z| < a \\
        0, \, \text{else} 
    \end{cases}
\end{align}
the monopole moment vanishes. The dipole moment is than given by
\begin{align}
    \phi_{\text{dip}}(r) &= \frac{a^2}{4 \pi \epsilon_0} \frac{\rho_0}{r^3} \int_{-1}^1 \text{d}\bar{x} \int_{-1}^1 \text{d}\bar{y} \int_{-1}^1 \text{d}\bar{z} \, x\cdot \bar{x}^2 \\
    &=\frac{a^5 \rho_0}{4 \pi \epsilon_0} \frac{8}{3 x^2} = \phi_0 \frac{8}{3x^2}.
\end{align}
Fig.\ref{fig:ex02_b} shows the potential inside, the potential outside and the analytically estimated potenial

\begin{figure}[h]
    \centering
    \includegraphics[width=\textwidth]{../code/build/ex02_b.pdf}
    \caption{Potential inside and outside the cube and the analytically estimated potential for $\rho_2$.}
    \label{fig:ex02_b}
\end{figure}